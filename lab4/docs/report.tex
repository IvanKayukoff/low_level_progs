\documentclass[listings]{labreport}
\departmentsubject{Кафедра вычислительной техники}{Языки системного программирования}
\titleparts{Лабораторная работа №4}{Вариант 1}
\students{Каюков Иван Алексеевич}

\begin{document}

\maketitlepage

\section*{Задание}

\textbf{Односвязный список:}

Программа должна принимать некоторое количество целых чисел с \textit{stdin}.
\begin{enumerate}
  \item Сохранить все числа в \textit{linked list} в обратном порядке.
  \item Написать функцию вычисляющую сумму всех элементов в 
    \textit{linked list}.
  \item Написать функцию для получения n-го элемента в списке. Если список
    слишком маленький - сообщить об этом.
\end{enumerate}

\textbf{Функции высших порядков:}

В \textit{stdin} подается некоторое количество целых чисел.
\begin{enumerate}
  \item Сохранить все числа в список.
  \item Реализовать функцию \textit{foreach}, используя ее вывести весь 
    список 2 раза: разделяя элементы списка пробелами, разделяя элементы 
    списка символом новой строки.
  \item Реализовать функцию \textit{map}, используя ее вывести квадраты и 
    кубы элементов списка.
  \item Реализовать функцию \textit{foldl}, используя ее вывести сумму 
    элементов списка, минимальное и максимальное значение в списке.
  \item Реализовать функцию \textit{map\_mut}, используя ее вывести модули 
    всех чисел в списке.
  \item Реализовать функцию \textit{iterate}, используя ее создать и вывести
    лист содержащий последовательность степеней двойки (первые десять
    значений).
  \item Реализовать функцию \textit{save}, которая будет записывать все 
    элементы списка в текстовый файл.
  \item Реализовать функцию \textit{load}, которая будет считывать все целые
    числа из текстового файла и помещать их в список.
  \item Реализовать функцию \textit{serialize}, которая будет записывать все 
    элементы списка в бинарный файл.
  \item Реализовать функцию \textit{deserialize}, которая будет считывать 
    все целые числа из бинарного файла и помещать их в список.
\end{enumerate}

\section*{Исходный код}

Исходный код доступен по адресу 
\texttt{https://github.com/IvanKayukoff/low\_level\_progs}

\section*{Вывод}

В данной лабораторной работе мы познакомились c простой
реализацией односвязного списка и функциями высших порядков на языке 
\textit{Си}.

\end{document}
