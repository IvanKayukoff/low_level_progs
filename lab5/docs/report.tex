\documentclass[listings]{labreport}
\departmentsubject{Кафедра вычислительной техники}{Языки системного программирования}
\titleparts{Лабораторная работа №5}{Поворот BMP изображения}
\students{Каюков Иван Алексеевич}

\begin{document}

\maketitlepage

\section*{Задание}

Реализовать программу на языке \textit{Си} поворачивающую BMP изображение 
любого разрешения на 90 градусов вправо или влево (реализовать необходимо
только поворот в одну из сторон).

\section*{Архитектура}

\begin{enumerate}
  \item Описать структура одного пикселя, чтобы не работать с растровым 
    массивом изображения как с абсолютно неразмеченными данными
  \item Отделить формат представления картинки внутри программы от формата
    BMP или любого иного
  \item Открыть bmp-файл и вернуть ошибку, если таковая произошла
\end{enumerate}

\section*{Исходный код}

Исходный код доступен по адресу 
\texttt{https://github.com/IvanKayukoff/low\_level\_progs}

\section*{Вывод}

В данной лабораторной работе мы познакомились c простой реализацией поворота
BMP картинки на языке \textit{Си}.

\end{document}
