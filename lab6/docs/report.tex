\documentclass[listings]{labreport}
\departmentsubject{Кафедра вычислительной техники}{Языки системного программирования}
\titleparts{Лабораторная работа №6}{Custom memory allocator}
\students{Каюков Иван Алексеевич}

\begin{document}

\maketitlepage

\section*{Задание}

Реализовать \textit{malloc} и \textit{free} с помощью системного вызова 
\textit{mmap} и связанного списка блоков произвольного размера.
Использовать \textit{malloc/calloc, realloc} и \textit{free} не разрешается.

\section*{Архитектура}

\begin{enumerate}
  \item Каждый блок содержит заголовок и данные, которые идут сразу после
    заголовка
  \item Заголовок состоит из указателя на следующий блок, размера данных и 
    флага, показывающего свободен ли этот блок
  \item Размер данных должен быть кратен \textit{CHUNK\_ALIGN}, т.е. 8-ми байтам
\end{enumerate}

\section*{Исходный код}

Исходный код доступен по адресу 
\texttt{https://github.com/IvanKayukoff/low\_level\_progs}

\section*{Вывод}

В данной лабораторной работе мы научились пользоваться \textit{mmap} и 
познакомились с реализацией \textit{malloc} и \textit{free} на языке \textit{Си}.

\end{document}
