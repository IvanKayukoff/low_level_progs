\documentclass[12pt, a4paper]{article}
\usepackage[a4paper, includeheadfoot, mag=1000, left=1.5cm, right=1.25cm, top=1.25cm, bottom=1.25cm, headsep=0.8cm, footskip=0.8cm]{geometry}

% Fonts
\usepackage{fontspec, unicode-math}
\setmainfont[Ligatures=TeX]{CMU Serif}
\setmonofont{CMU Typewriter Text}
\usepackage[english, russian]{babel}

% Include code
\usepackage{listings}

% Indent first paragraph
\usepackage{indentfirst}
\setlength{\parskip}{5pt}

% Diagrams
\usepackage{graphicx}
\usepackage{float}
\usepackage{subcaption}

% Page headings
\usepackage{fancyhdr}
\pagestyle{fancy}
\renewcommand{\headrulewidth}{0pt}
\setlength{\headheight}{16pt}

\begin{document}

\begin{titlepage}
\begin{center}

\textsc{«УНИВЕРСИТЕТ ИТМО»\\[4mm]
Кафедра вычислительной техники}
\vfill
\textbf{ЛАБОРАТОРНАЯ РАБОТА \textnumero 3}\\[4mm]
по дисциплине:\\[2mm]
\textbf{Языки системного программирования}\\[16mm]
\end{center}

\begin{flushright}
Выполнил: \\[4mm]
Каюков Иван Алексеевич \\[2mm]
Группа P3202
\end{flushright}

\begin{center}
\vfill
Санкт-Петербург\\[2mm]
2018 г.

\end{center}
\end{titlepage}


\textbf{Задание:} реализовать 2 функции на \textit{Си}, используя стандарт
\textit{-std=c99}.
\begin{enumerate}
\item Реализовать функцию вычисляющую скалярное произведение двух
массивов, принятых как аргументы.
\item Реализовать функцию проверяющую переданное число на простоту.
\end{enumerate}

\textbf{Реализация:}

\lstinputlisting[firstline=3, language=C]{../main.c}

\textbf{Вывод:} в данной лабораторной работе мы познакомились с основами
языка программирования \textit{Си}.

\end{document}

